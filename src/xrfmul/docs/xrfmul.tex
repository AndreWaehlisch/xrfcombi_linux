\documentstyle[12pt]{article}
\title{GEBRUIK XRFMUL}
\author{M.Bos}
\date{7 december 1999}
\begin{document}
\maketitle
\section*{Algemeen}
Het programma {\bf xrfmul} kan voor meerlaags preparaten uit 
gemeten intensiteiten van r\"ontgenfluorescentielijnen de dikte van
de lagen en hun samenstelling berekenen. Voorwaarde daarbij is
dat er om te beginnen voldoende lijnen zijn gemeten en verder
natuurlijk dat ook de elementen in de onderste laag nog bijdragen
aan de gemeten intensiteiten. Sommige auteurs beweren dat de lagen
geen gemeenschappelijke elementen mogen bezitten, maar dat is mijns inziens
geen wet van Meden en Perzen.

\section*{Benodigde bestanden}
Het programma {\bf xrfmul} haalt zijn benodigde invoergegevens uit bestanden.
De vaste gegevens dienen als {\bf .dat} en {\bf .txt} bestanden in de
werkdirectory van {\bf xrfmul} te staan. De monster-afhankelijke
gegevens moeten worden ondergebracht in de volgende bestanden:
\begin{itemize}
\item pwsettings.dat -- de gegevens van de gebruikte  spectrometer
\item lines.dat -- de gemeten lijnen met hun relatieve intensiteiten (t.o.v.
het zuivere element)
\item sample.dat -- bekende gegevens van het monster; in ieder geval
de kwalitatieve opbouw van de verschillende lagen
\end{itemize}
De exacte layout en inhoud van deze bestanden wordt hieronder beschreven.
Zij dienen zich te bevinden in de subdirectory {\bf ./data}.

\subsection*{Het bestand pwsettings.dat}
Dit bestand
is \'e\'enregelig en bevat achetreenvolgens : symbool van het anode-element,
de dikte in g/cm$^2$ van het berylliumvenster van de r\"ontgenbuis,
de take-off angle van de straling bij de anode, het getal 1.0,
de incidence angle bij het monster (in graden) en de take-off angle
bij het monster (ook in graden).
Voor onze spectrometer met de chroombuis dient dit bestand er als volgt uit
te zien:
\begin{verbatim}
Cr 0.0925001 26.0 1.0 61.0 40.0
\end{verbatim}
\subsection*{Het bestand lines.dat}
Het bestand {\bf lines.dat} bevat de gegevens omtrent de gemeten lijnen.
Per regel worden de gegevens van \'e\'en lijn genoteerd en wel in de
volgorde: symbool van het element, aanduiding van de lijn,
het kilovoltage waarbij gemeten is en de gemeten relatieve intensiteit.
Een voorbeeld van zo'n bestand:
\begin{verbatim}
Si ka 50.0  4.309358e-2
Ti ka 50.0 3.907203e-2
As ka 50.0 3.792139e-2
Ag ka 50.0 1.375333e-2
\end{verbatim}
\subsection*{Het bestand sample.dat}
De eerste regel van dit bestand bevat \'e\'en getal dat aangeeft
uit hoeveel lagen het preparaat bestaat. Daarna volgt er per laag
een regel met de gegevens van die laag. De tweede regel van het bestand
bevat de gegevens van de bovenste laag van het preparaat (gezien vanaf
de detector en de r\"ontgenbuis).

De regels met de gegevens van een laag hebben de volgende structuur:
geheel getal dat het aantal componenten in die laag representeert.
Vervolgens een beginschatting of de werkelijke waarde van de 
dikte in g/cm$^2$ van de laag gevolgd door het getal 0 of het getal 1.
Het getal 1 geeft aan dat deze parameter niet "opgeknapt" hoeft te
worden in de iteratieprocedure en dus de werkelijke dikte representeert.
Wordt hier het getal 0 gebruikt dan wordt de dikte tijdens de 
iteratieve berekening aangepast.
 
Hierna volgen per component de naam van de component, een schatting van
 de gewichtsfractie
van die component in de laag of de werkelijke waarde van die gewichtsfractie,
ook weer gevolgd door een 0 of een 1 met dezelfde betekenis van resp. opknappen
of vasthouden.

Een voorbeeld van het bestand {\bf sample.dat}:
\begin{verbatim}
2
3 6.96e-4 0 Ti 0.2005 0 As 0.400039 0 Ag 0.4011 0
1 8.474e-3 0 Si 1.0 1
\end{verbatim}
Het preparaat dat met dit bestand wordt gekarakteriseerd bestaat uit
2 lagen, de bovenste laag bevat 3 componenten, heeft een beginschatting
voor de dikte van 6.96e-4 g/cm$^2$ die moet worden opgeknapt,
evenals de fracties 0.2005 0.400039 en 0.4011 van resp. de elementen
Ti, As en Ag.

De onderste laag bestaat uit zuiver Si met een fractie 1.0,
 die vast moet blijven
staan. Wel moet voor de onderste laag de beginschatting 8.474e-3 g/cm$^2$
van de dikte worden opgeknapt.

\section*{Het berekenen van de onbekende gegevens van het preparaat}
Na het aanmaken van bovengenoemde bestanden in de subdirectory {\bf ./data}
kan de berekening worden gestart door in de werkdirectory
het commando:
\begin{verbatim}
xrfmul
\end{verbatim}
te geven.

Een voorbeeld van de resultaten:
\begin{verbatim}
0 Z: 14 line ka rel int. 0.0430936 
1 Z: 22 line ka rel int. 0.039072 
2 Z: 33 line ka rel int. 0.0379214 
3 Z: 47 line ka rel int. 0.0137533 
Bulkint Si ka  1.323464e-04 
Bulkint Ti ka  1.378835e-03 
Bulkint As ka  4.437061e-04 
Bulkint Ag ka  3.405200e-04 
Warning maximum # iterations reached 
Line nr 0 RXI found 0.0428934 RXI meas 0.0430936
Line nr 1 RXI found 0.0390286 RXI meas 0.039072
Line nr 2 RXI found 0.0375672 RXI meas 0.0379214
Line nr 3 RXI found 0.0136809 RXI meas 0.0137533
Layer 0 has 3 compounds and a massthickness of 0.000629728  g/cm2
Compound Ti wfract 0.123567  
Compound As wfract 0.445756  
Compound Ag wfract 0.430676  
Layer 1 has 1 compounds and a massthickness of 0.0575096  g/cm2
Compound Si wfract 1  
Ssq is 0.000137984
\end{verbatim}
\end{document}
